\documentclass[conference]{IEEEtran}

\IEEEoverridecommandlockouts
\usepackage{cite}
\usepackage{amsmath,amssymb,amsfonts}
\usepackage{algorithmic}
\usepackage{graphicx}
\usepackage{textcomp}
\usepackage{xcolor}
\def\BibTeX{{\rm B\kern-.05em{\sc i\kern-.025em b}\kern-.08em
    T\kern-.1667em\lower.7ex\hbox{E}\kern-.125emX}}
\begin{document}

\title{Omnetpp dan Algoritma Dijkstra}

\author{\IEEEauthorblockN{Fajar Bimantara}
\IEEEauthorblockA{\textit{Institut Teknologi Batam} \\
\textit{Teknik Komputer}\\
Batam, Indonesia \\
1922027@student.iteba.ac.id}
\and

\IEEEauthorblockN{Muhammad Riyadhtul Akbar}
\IEEEauthorblockA{\textit{Institut Teknologi Batam} \\
\textit{Teknik Komputer}\\
Batam, Indonesia \\
1922011@student.iteba.ac.id}
\and
}
\maketitle

\begin{abstract}
    Algoritma Dijkstra merupakan algoritma yang dapat digunakan dalam
    pencarian lintasan terpendek,di mana memiliki iterasi untuk mencari titik yang jaraknya
    dari titik awal adalah paling pendek. Pada setiap iterasi, jarak titik yang diketahui (dari titik
    awal) diperbarui bila ternyata didapat titik yang baru yang memberikan jarak terpendek.
\end{abstract}
\begin{abstract}
Pengenalan OMNet++ dan Bahasa NED
adalah sebuah framework simulasi jaringan discrete-event yang bertipe object-oriented. Terlalu panjang? Baiklah, akan kita sederhanakan pengertian ini.
Simulator jaringan discrete-event berarti simulator tersebut bertindak/bereaksi atas kejadian-kejadian yang berlangsung di dalamnya (event). Secara analitis, jaringan komputer adalah sebuah rangkaian discrete-event. Komputer akan membuat sesi memulai, sesi mengirim dan sesi menutup. OMNet++ bersifat object-oriented berarti setiap peristiwa yang terjadi di dalam simulator ini berhubungan dengan objek-objek tertentu.
OMNet++ juga menyediakan infrastruktur dan tools untuk memrogram simulasi sendiri. Pemrograman OMNet++ bersifat object-oriented dan bersifat hirarki. Objek-objek yang besar dibuat dengan cara menyusun objek-objek yang lebih kecil. Objek yang paling kecil disebut simple module, akan memutuskan algoritma yang akan digunakan dalam simulasi tersebut.
\end{abstract}
\begin{IEEEkeywords}
    Algoritma Dijkstra, Lintasan Terpendek, Web
\end{IEEEkeywords}

\section{Introduction}
Algoritma Dijkstra merupakan salah satu algoritma yang efektif dalam memberikan
lintasan terpendek dari suatu lokasi ke lokasi yang lain. Prinsip dari algoritma Dijkstra adalah
dengan pencarian dua lintasan yang paling kecil.Algoritma Dijkstra memiliki iterasi untuk
mencari titik yang jaraknya dari titik awal adalah paling pendek.

Dalam rincian prosedur tersebut v0 merupakan titik awal yang ditentukan. D[v] adalah
jarak terpendek dari v0 ke titik v. C adalah matrik bobot, dan C[w , v] adalah jarak (bobot) dari
titik w ke titik v. Min adalah fungsi untuk mencari nilai terkecil dari dua nilai, himpunan S
digunakan untuk mencatat titik-titik yang terpilih pada setiap iterasi dan himpunan V berisi
semua titik dalam graf.Menurut Aldous dan Wilson(2000: 6), graf merupakan diagram yang
memuat titik yang disebut verteks dan dihubungkan oleh garis yang disebut sisi, dengan
masing-masing sisi tepat menghubungkan dua titik. 

\section{Scenario}
\subsection{METODE}
Penelitian ini didasarkan studi literatur dari hasil penelitian yang telah dilakukan
untuk dikembangkan lebih lanjut. Menurut Sukmadinata (2009: 172) pada tahap studi literatur
ini dilakukan kajian literatur secara intensif dengan menggali konsep-konsep atau teori-teori
yang mendukung dalam pembuatan produk program yang dibuat. Studi literatur ini sangat
penting dilakukan terutama dalam pembentukan suatu produk yang berbentuk software karena
dalam pembentukannya diperlukan dasar-dasar konsep dan teori tertentu. Dengan kegiatan
studi literatur ini dikaji luasan produk, teknik pembuatan dan implementasi hasil produk agar
dapat digunakan secara maksimal. 

\section{Hasil Dan Pembahasan}
Dalam perancangan pembuatan program implementasi algoritma Dijkstra dalam
pencarian lintasan terpendek lokasi rumah sakit, hotel dan terminal kota Malang berbasis web
menggunakan beberapa piranti keras dan piranti lunak sebagai berikut :
\begin{itemize}
    \item Komputer dengan processesor Intel Pentium P600
    \item RAM 2 GB
    \item Harddisk 320 GB
    \item Windows 7 Ultimate
    \item Adobe Dreamweaver CS3
    \item XAMPP versi 2.5
  \end{itemize}
\section{Kesimpulan}
Algoritma Dijkstra Dalam Pencarian Lintasan
Terpendek Lokasi Rumah Sakit, Hotel dan Terminal Kota Malang Berbasis Web dilakukan
dengan mengkonversi prosedur algoritma Dijkstra menjadi script program kemudian
disertakan dalam web.

\bibliographystyle{IEEtran}
@article{sholichin2012implementasi,
  title={Implementasi Algoritma Dijkstra Dalam Pencarian Lintasan Terpendek Lokasi Rumah Sakit, Hotel Dan Terminal Kota Malang Berbasis Web},
  author={Sholichin, Riyadhush and Yasindan, M and Oktoviana, LT},
  journal={Jurnal Online Universitas Negeri Malang,(Online)},
  year={2012}
}
\bibliography{referensi.bib}

\end{document}
